% Введение
\section*{Введение}

\begin{frame}
    \frametitle{Введение}
    \begin{center}
          \begin{itemize}
            \item \textbf{Алюминий (Al)} — лёгкий, серебристо-белый металл (атомный номер 13), третий по распространённости в земной коре, используется в промышленности благодаря низкой плотности, высокой электропроводности и коррозионной стойкости
            \item \textbf{В экономическом смысле} это стратегический металл, ключевой для транспорта, строительства, электроники и упаковки. Его переработка и производство требуют больших энергозатрат, а растущий спрос связан с переходом к устойчивой экономике и индустриализацией в развивающихся странах
            \item \textbf{Код ТН ВЭД:} 76.XX.XX
            \item \textbf{Товары-субституты:} медь, сталь, никель, титан, легкие пластики и углеродные композиты, магний
            \item Металлы, включая алюминий, остаются востребованными в связи с энергетическим переходом и ростом спроса на легкие и устойчивые материалы. Однако высокие затраты на производство и потенциальные геополитические риски могут ограничить их доступность в краткосрочной перспективе
          \end{itemize}
    \end{center}
\end{frame}