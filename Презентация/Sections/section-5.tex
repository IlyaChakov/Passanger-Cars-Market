% Регулирование
\section{Регулирование}

\begin{frame}
    \frametitle{Тарифные меры}
    \begin{itemize}
        \item \textbf{США:} В марте 2018 года администрация президента Дональда Трампа ввела пошлины в размере 25\% на импорт стали и 10\% на импорт алюминия из ряда стран, включая Китай. В феврале 2023 года США повысили пошлину на импорт алюминия из России до 200\%, что фактически прекратило поставки российского алюминия на американский рынок
        \item \textbf{Европейский Союз:} ЕС рассматривает возможность введения санкций на импорт алюминия из России, что может привести к конкуренции с США за поставки металла из других регионов и усилению инфляции
        \item \textbf{Канада:} В декабре 2023 года Канада ввела запрет на импорт алюминия и стали из России
    \end{itemize}
\end{frame}

\begin{frame}
    \frametitle{Нетарифные меры}
    \begin{itemize}
        \item \textbf{США и Мексика:} В июле 2024 года США и Мексика договорились об усилении контроля за происхождением стали и алюминия, импортируемых через мексиканские порты, чтобы предотвратить обход существующих тарифов
        \item \textbf{Европейский Союз:} ЕС может расширить антироссийские санкции на импорт алюминия, что приведет к необходимости поиска альтернативных поставщиков и возможному росту цен на металл.
    \end{itemize}
\end{frame}