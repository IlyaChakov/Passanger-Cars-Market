% Методология исследования
\section{Методология исследования}

\begin{frame}[fragile]
    \frametitle{Методология исследования}
    \begin{columns}[t]
        % Первая колонка (текст)
        \begin{column}{0.5\textwidth}
            \textbf{Проделанная работа:}
            \begin{itemize}
                \item Пропущенные значения ддатафреймов и значения будущих периодов были спрогнозированы на основе \textbf{ETS-модели} временных рядов
                \item В целях оптимизации исследования одним из авторов был написан скрипт в $R$ в виде функции для обработки датафреймов (листинг справа, полная версия — на репозитории GitHub)
            \end{itemize}
        \end{column}

        % Вторая колонка (изображение)
        \begin{column}{0.5\textwidth}
            \textbf{Листинг функции:}
            \begin{verbatim}
result_df <- fill_forecast_ets(
    data = df, # Исходный датафрейм
    time = "Date", # Временная шкала
    frequency = 1, # Сезонность
    cols = NULL, # Фиксированные столбцы
    n_periods = 0 # Периоды прогноза
)
            \end{verbatim}
        \end{column}
    \end{columns}
\end{frame}